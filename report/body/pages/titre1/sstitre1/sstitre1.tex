\subsection{Teoria de Aprendizagem}

Aquisição de Linguagem é o processo pelo qual seres humanos tornam-se capazes de compreender e reproduzir uma língua. Esse mecanismo, de um ponto de vista analítico, pode ser desenhado em torno de questões de \emph{o quê}, \emph{quando} e \emph{como} um indivíduo desenvolve uma determinada habilidade lingüística \citep{pearl2010}.\\

Então, seria possível organizar, de maneira simplificada, um fluxo lógico dos processos que levariam à aquisição de uma língua nas perguntas abaixo.

\begin{itemize}
\item Qual é o objeto a ser aprendido?
\item Quando o objeto é aprendido?
\item Como o objeto é aprendido?
\end{itemize}

Pensando na língua mãe, o objeto em questão é o conteúdo lingüístico que a criança adquire, isto é, as palavras e os fonemas e, junto com eles, a carga semântica e regras sintáticas. O próximo passo é analisar o momento em que cada habilidade é dominada e estabelecer uma seqüência ordenada de acontecimentos, pontuando na linha do tempo marcas de proficiência com a língua. Por exemplo, a capacidade de discernir palavras na fala é obviamente anterior à capacidade de conjugar ações no passado.\\

Contudo, dissecar exatamente o desenrolar dessa etapas, isto é, \emph{como} a criança aprende \emph{o que} até o tempo de \emph{quando}, é mais intrincado.\\

Se postos lado a lado, um bêbe humamo e um chimpanzé, assusta-se perceber que o segundo vai nos superar em muitas tarefas. Sobretudo, em se tratando de memorização. Incrivelmente, poucos anos depois de dominar inteiramente 500 mil vocábulos.





