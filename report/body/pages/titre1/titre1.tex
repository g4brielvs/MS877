
\section{Da Filosofia à Linguística}

Lingüística é a Ciência por trás da compreensão da forma e estrutura das línguas humanas. Não há quem não tenha curiosidade 
pelo que é falado. É um fenômeno que nos parece tão natural e simples, mas quando se reflete sobre ela o que se encontra é uma complexidade que fascina qualquer cientista ou curioso. Talvez por essa razão a questão da origem da fala tenha permeado tão constantemente o imaginário mítico. É curioso que muitos mitos coincidem ao dizer que a linguagem é uma dávida divina. A capacidade de se comunicar abstratamente também é um importante fator nos distingue de todos outros seres vivos que habitam o planeta: capacidade de aprender e ensinar através da língua. 

Nessa primeira seção, será tratado o percurso que levou o estudo das línguas evoluir de uma esfera de especulação do pensamento filosófico para um exercício científico. Muito foi descoberto a respeito da natureza dos processos cognitivos, isto é, se era uma discussão reservada para a abstração, experimentações passaram a mostrar como mensurar fenômenos da mente.

\subsection{Filosofia da Mente}

O problema Mente-Cérebro é a dicotomia entre mundo interior e o mundo físico na experiência da consciência. Existem evidências antropológicas que sustentam a crença de que os primeiros seres humanos já tinham o entendimento da diferença entre corpo e alma. Para os filósofos da Antigüidade, era de conhecimento geral ser o cérebro o órgão responsável pelos sentimentos, sentidos e inteligência. No Renascimento, com o florescer das Ciências Naturais, muito foi descoberto nas áreas de Anatomia e Fisiologia. Porém, a questão da origem e manifestação da consciência ainda estava em aberto. 
Uma das primeiras grandes influências modernas para responder ao problema foi de René Descartes. Sua teoria propunha a existência de uma componente imaterial, chamada de \emph{res cogitans}, em oposição a uma componente material chamada de \emph{res extensa}. Esses dois componentes teriam papéis iguais para construção da consciência e interagiriam entre si no interior de estruturas cerebrais. Essa corrente ficou conhecida por Dualismo. Graças a essa forma de pensar, pouco a pouco o estudo da mente começou a ganhar o domínio mais exato nos campos da Fisiologia e Medicina. Esse avanço deu nascimento à Psicologia Experimental.

\subsection{Psicologia Experimental}

Diante do florescer das Ciências Naturais na Revolução Científica durante os séculos XVII  XVIII, grandes cientistas contribuíram para o entendimento da mente pelo viés empírico e epistemológico. Graças à invenção do microscópio e descobertas no estudo de anatomia humana, os conhecimentos sobre o sistema nervoso foram consolidados e exploradas suas relações com outros sistemas do corpo. 

No século XIX, Hermann Helmholtz foi responsável em calcular a velocidade da condução nervosa. Esse resultado teve uma forte influência para Psicologia Científica ao constatar que a comunicação através de nervos não é instantânea e ocorre via excitação elétrica. Porém, somente anos mais tarde que vieram as bases do que hoje é chamada Psicologia.

De um ponto de vista quantitativo, o primeiro cientista a se dedicar a estudos psicológicos foi o alemão Ernst Weber. No início do século XIX, seu interesse acadêmico o pôs a explorar a fisiologia por trás da resposta a estímulos externos, sobretudo, envolvendo a sensação de tato. Foi o responsável por realizar os primeiros estudos de limiares. Sua contribuição principal foi na percepção de pressão sobre a pele. Ele propôs um experimento para avaliar em que momento é percebida uma diferença de pressão. Em teste cego, voluntários comparam um peso padrão com pesos de teste, pedindo-se para discriminar o mais leve ou o mais pesado.  A conclusão, conhecida por Lei de Weber, é de que não se percebe a diferença entre os objetos, mas a proporção dessa diferença com a magnitude dos objetos. Portanto, a lei de Weber expressa a diferença necessária para perceber o estímulo como diferente. 

Em seguida, Gustav Fechner, tendo ciência dos trabalhos de Weber, aprimorou o método dos limiares e estudo de percepções. No seu entendimento, as diferenças refletiam não somente uma diferença quantitativa, mas uma diferença psicológica. Em outras palavras, a percepção de estímulo, mesmo podendo ser mensurado, tem inerentemente um caráter subjetivo. Fechner também foi responsável por executar experimentos com estímulos visuais e sonoros, chegando a conclusões semelhantes.
Por sua vez, coube ao alemão Wilhelm Wundt a formalização desses esforços na forma de uma nova ciência. Sendo fundador do primeiro laboratório de Psicologia, foi o pioneiro em experimentação com a mente e a consciência. Wundt estabeleceu a distinção entre psicologia experimental, sujeita a métodos científicos mais tradicionais e trabalhos de laboratório, e psicologia cultural, que caberia a um escopo mais filosófico. Além disso, criou uma metodologia clara fazendo a distinção de faculdades elementares, passíveis de reprodução em laboratório, e faculdades superiores, tais como linguagem. Para Wundt, a consciência não se trata de moralidade, mas de uma consciência perceptual. Wundt foi um dos pesquisadores mais produtivos, escrevendo, em média, três páginas por dia ao longo de sua vida.

O cientista James Cattell foi o primeiro americano a receber o título de doutor sob a orientação de Wundt. Muito influenciado pela metodologia das escolas alemãs, Cattell foi o principal incubido de desenvolver e concretizar a visão de Wundt. Cattell realizou experimentos para obter tempo de resposta introduzindo complexidade nas tarefas. Por exemplo, em um de seus ensaios, na primeira etapa pede-se a um voluntário que somente indique quando perceber o acendimento de um sinal luminoso. Na segunda etapa, pede-se que indique somente quando o sinal for de uma determinada cor. A diferença de tempo deve representar o tempo de processamento no cérebro.  Assim, o cientista coletou as primeiras medidas mais assertivas de cronometria mental. Cattell foi um pioneiro em experimentos envolvendo linguagem. 

Nesse contexto, a linguagem passou a ser estudado como um processo cognitivo como qualquer outro, passível de experimentação.

\subsection{Teoria Cognitivista}

No início do século XX, a Psicologia foi fortemente influenciada pelos trabalhos Skinner e outros cientistas behavioristas. Nessa corrente científica, colocava-se como objeto de estudo o comportamento. Para Skinner, a mente tem uma natureza intangível e não-observável, cabendo então à pesquisa científica somente estudar o comportamento. Esse período ficou conhecimento como a fase em que "Psicologia perdeu a cabeça". Os resultados relacionados a condicionamento de comportamento e aprendizado tiveram grande influências nos métodos militares e escolares usados até hoje em dia. Muito também foi descoberto a respeito do comportamento animal e quão semelhante é comparado ao ser humano.

Entretando, anos mais tarde nos anos 1950, esforços de diferentes áreas, desde Ciência da Computação até Psicologia, argumentavam que essa abordagem não seria suficiente para explicar todos fenômenos cognitivos. O principal argumento era que o modo de pensar seria instrinsecamente um processo da mente e afetaria o comportamento, sendo assim, inadequado se limitar ao estudo da conseqüência, isto é, do comportamento por si só. Esse período ficou conhecido por Movimento Cognitivista. Apesar de trazer de volta para o holofote o estudo a mente, a abordagem cognitivista faz fortemente uso de métodos quantitativos, buscando descrever as funções da mente e o processamento das informações. Particularmente na Lingüística, houve uma grande contribuição para nascimento da Gramática Generativa. O linguista norte-americano Noam Chomsky \citep{chomsky65} propôs a capacidade da linguagem pode ser vista como uma gramática constituída por um sistema de regras para combinação de palavras e formação de sentenças. Nesse contexto, adquirir uma linguagem equivale a ganhar domínio sobre esse sistema de regras.






