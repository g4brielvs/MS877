
\subsection{Abordagem computacional}

O principal benefício de uma abordagem computacional é que se trata de uma simulação, assim sendo possível ter controle e flexibilidade sobre as condições que se deseja experimentar. 


Falar do \citep{marr1982vision}

\begin{itemize}
\item Computacional
\item Algorítmico
\item Implementacional
\end{itemize}



\subsection{Abordagem com aprendizado de máquina? Não!}

Em 1997, um ser humano foi vencido por um computador em uma partida de xadrez \footnote{\url{http://www-03.ibm.com/ibm/history/ibm100/us/en/icons/deepblue/}}. O super computador Deep Blue superou somente com circuitos e algoritmos um dos limites de inteligência humana. Mentira. Foram necessários mais de cinqüenta anos de pesquisa, engenho e sorte. Mas, no final das contas, fomos vencidos. Em 2017, Google realizava testes com um veículo autônomo \footnote{\url{http://www.google.com/selfdrivingcar/}}, isto é, sem a necessidade de um motorista humano. Novamente, fomos vencidos.\\

Agora, basta ter uma conversa com a Siri ou a Alexa para perceber que ninguém jamais aceitaria um convite para jantar de qualquer uma delas. Embora rendam boas piadas, os seus mecanismos de fala são muitos artificiais. O que torna a capacidade falar tão inerentemente humana? E o que torna falar tão intricamente difícil? Aliás, uma criança capaz de falar com naturalidade, apesar de não fazer ideia o que fazer com uma alavanca do câmbio de marchas.


