\section*{Comentários Finais}
\addcontentsline{toc}{section}{Comentários Finais}

O estudo da aquisição de linguagem mostrou-se um campo frutífero e intersdisciplinar. A linguagem sempre suscitou a curiosidade filosófica e científica, tendo seu entendimento um longo histórico e com grande evolução até o presente momento. Desde do estabelecimento como uma ciência, a Lingüística e, mais particularmente a Aquisição da Linguagem, ganhou técnicas e métodos para pesquisa e análise. Os experimentos precursores durante os séculos XVIII e XIX deram a base para a formalização das teorias cognitivistas. Durante o século XX, o mundo assistiu um movimento a caminho da estruturação das Ciências Humanas e, em conseqüência, os lingüistas receberam contribuições de outras disciplinas e de outras teorias.

A partir desse momento, a Teoria de Aprendizagem e Inteligência Computacional muniram o estudo da aquisição de linguagem de métodos quantitativos e matemáticos. Essa união teve resultados produtivos. Primeiramente, ao se criar uma representação de um fenômeno lingúístico, é preciso ter consciência das variáveis livres e dependentes e quais são as suas relações, o que contribui para o entendimento do problema como um todo. Além disso,  usando uma simulação computacional, é possível testar e levantar hipóteses com grande facilidade, não se limitando a condições físicas ou sociais. É importante lembrar que a Ciência da Computação e Lingüística se complementam em uma via de mão dupla. Da mesma maneira que máquinas auxiliem no estudo da linguagem, as teorias lingüísticas dão a base para a existência de inúmeras aplicações, por exemplo, correções de texto, traduções e transcrições.