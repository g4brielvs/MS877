
\selectlanguage{portuguese}
\begin{abstract}

A capacidade de linguagem é uma das maiores vantagens evolutivas dos seres humanos. Comunicação verbal e escrita são a maneira fundamental pela qual é possível passar informação e conhecimento para novas gerações e resolver problemas abstratos e complexos em uma sociedade. Apesar de crianças começarem a falar com grande naturalidade, ainda há muito a se entender sobre a aquisição de linguagem e o tema ocupa um papel central na pesquisa científica na Lingüística. Esse relatório tem o objetivo de introduzir a Aquisição da Linguagem do ponto de vista da Gramática Generativa. Em primeiro momento, é apresentado um breve do contexto histórico e, em seguida, é introduzido um breve \emph{framework} teórico. Em segundo momento, são mostrados pontos positivos e negativos de abordagens computacionais para a Aquisição da Linguagem.

\end{abstract}

{\bf Keywords: Lingüística Computacional, Aquisição de Linguagem}

\newpage

\selectlanguage{english}
\begin{abstract}

The capacity for language is one of mankind's most important evolutionary advantages. Verbal and written communication 
is the key to our ability to pass information and knowledge forward to younger generations and our capacity to address abstract and complex problems as a society. Even tough a child can easily start talking, language acquisition remains an intricate subject and is at the heart of scientific inquiry and intellectual curiosity in Linguistics. This report is an introduction to the study of Language Acquisition in the perspective of generative grammar. At first, it introduces the reader to a quick historical background and brief theoretical framework. At second, it summarizes positives and negatives perspectives about a computational approach to Language Acquisition.

\end{abstract}

{\bf Keywords: Computational Linguistics, Language Acquisition}

\newpage
