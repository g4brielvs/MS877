\section*{Introdução}
\addcontentsline{toc}{section}{Introdução}
\noindent
Quando se aborda com ingenuidade a aquisição de fluência de uma língua, o primeiro pensamento que ocorre é que esse processo seja supostamente trivial. Aliás, crianças na mais tenra idade o fazem com grande facilidade. Porém, qualquer curioso que se debruce sobre o problema será confrontado com percalços e paradoxos, desde o que significa a rigor aprender uma língua e, antes de tudo, a definição do que é uma língua.

Os estudos dirigidos tiveram o objetivo de introduzir o grande \emph{framework} teórico por trás da Aquisição de Linguagem. Em primeiro lugar, com intuito de contextualizar historicamente, é apresentado um resumo de como as teorias evoluíram de especulações filosóficas para um formato científico. A questão da origem da fala sempre permeou o imaginário mítico, comumente sendo considerada uma dádiva divina, teve historicamente uma aproximação maior com questões filosóficas que científicas e só ganhou o patamar de exploração científica com os esforços durante as mudanças do florescer das Ciências nos séculos XVIII e XIX.

Em seguida, é explorado o problema de aquisição de linguagem de um ponto de vista da Lingüística. Nesse segmento, são definidos os conceitos de competência e performance lingüísticas, quais seriam as qualidades necessárias para adquirir uma língua e a caracterização como processo biológico.

Por fim, são introduzidos os principais argumentos que sustentam a abordagem da aquisição de linguagem com base na Teoria de Aprendizagem e Gramática Generativa. Finalmente indicamos de que maneira abordagens computacionais têm a contribuir para atacar a solução do problema, mostrando diferentes pontos positivos e negativos. São discutivos os modelos generativo, estatístico, social e espacial e desenvolvimentista.