
\section{Aquisição de Linguagem}

A Aquisição da Linguagem é para muitos um dos problemas centrais da Lingüística. Porém, o estudo dos processos relacionados à capacidade de linguagem e inteligência não são exclusivos aos lingüistas. Como já mencionado, a origem das línguas sempre permeou a imaginação coletiva, sendo um frutífero campo para ficção e filosofia. Além disso, a habilidade de falar e conceber e comunicar ideais é compreendida como um grande indicativo de consciência. Curiosamente, as primeiras tentativas na direção do estabecimento mais formal do estudo da linguagem vêm dos esforços da Matemática e Inteligência Artificial. O matemático Alan Turing \citep{turing1950computing} já levantava a questão se máquinas seriam capazes de pensar. Turing sugeriu trocar as palavras \emph{pensar} e \emph{máquina} por um critério mais claro e concreto. A ideia era testar se uma máquina é capaz de demonstrar comportamento inteligente. Hoje é conhecido como Teste de Turing. Analogamente, na área da Teoria  da Aprendizagem, busca-se definir com rigor matemático o que significa adquirir a capacidade de linguagem ou, em outras palavras, adquirir e dominar competências lingüísticas. Logo, no presente contexto, a aquisição de linguagem será vista como processos cognitivos e lingüísticos que permitem o falante de se expressar em uma língua, tendo como base a teoria da Gramática Generativa.

\subsection{Competência Lingüística}

Um falante nativo é capaz de avaliar se frases fogem a uma regra de sua língua ainda que não saiba a razão específica da suspeita. Hipoteticamente, se lhe fosse pedido registrar todo seu conhecimento, dar-se-ia conta de não ter todo conteúdo em nível consciente, apesar de ser capaz de se comunicar fluentemente. Essa habilidade intrínseca e inconsciente é sua competência na língua. Já sua real capacidade de articular e por em prática uma linguagem é chamada de performance.
 A distinção entre as duas é bem nítida para qualquer falante, uma vez que é comum se atrapalhar para construir frases, parar no meio e reformular, errar conjugação de um verbo e não é por isso que não se assume conhecimento lingüístico. Essa distinção é um importante marco teórico \citep{chomsky65} para salientar a diferença entre não ter conhecimento pleno e somente cometer um deslize de fala na produção da língua. \\
 
Em suma, falar uma língua implica ter o domínio de diferentes níveis lingüísticos, isto é, 
ter a habilidade de usar diferentes qualidades relacionadas à linguagem. No quadro abaixo está sumarizado o conjunto dessas qualidades e suas respectivas áreas de estudo dentro da Lingüística.
 
\begin{center}
\begin{tabular}{| l | l | l |}
 \hline 
 \textbf{Constituintes} &  \textbf{Capacidade} & \textbf{Área} \\ 
 \hline 
 Sons & Fonemas & Fonética/Fonologia \\ 
 \hline 
 Palavras & Léxico & Morfologia \\ 
 \hline 
 Sentenças & Combinação de palavras & Sintaxe \\ 
 \hline 
 Conceitos & Sentido das senteças & Semântica \\ 
 \hline 
 Intenções & Propósito da fala & Pragmática \\ 
 \hline 
 \end{tabular}  
 \end{center}
 
Na investigação lingüística, mais particularmente na Aquisição de Linguagem, o que se propõe é elucidar como cada qualidade está relacionada ao processo de aquisição de linguagem e construção da gramática do falante. Abaixo, temos questões que cada área da Lingüística deveria ser capaz responder ao passo que um falante o faz natural e inconscientemente. \\
 
 \begin{itemize}
 \item Fonética
 	 \begin{itemize}
 		\item Quais são os sons elementares de uma linguagem?
 		\item Como segmentar e combinar sons em palavras?
 	\end{itemize}
 \end{itemize}

\begin{itemize}
 \item Morfologia
 	 \begin{itemize}
		\item Quais são as unidades de sentido (palavras)?
		\item Como identificar segmentação entre palavras na fala?
		\item Como identificar a quais pessoas, objetos ou eventos as palavras se referem?
		\item Como construir vocabulário funcional?
 	\end{itemize}
 \end{itemize}
 
 \begin{itemize}
 \item Sintaxe
 	 \begin{itemize}
		\item Como combinar palavras para construir sentenças que carregam sentido?
		\item Qual é a hierarquia de palavras de uma sentença?
		\item Como classificar uma sentença como gramatical?
 	\end{itemize}
 \end{itemize}

 \begin{itemize}
 \item Pragmática
 	 \begin{itemize}
		\item Como demonstrar intenção?
 	\end{itemize}
 \end{itemize}

\subsection{Processo Biológico}
\label{sec:processobiologico}

A aquisição de linguagem antes de tudo é tida como um processo biológico, similar de outros órgãos do corpo, pela perspectiva chomskiana. Uma série de evidências reforçam a proposição de se tratar de um processo biológico, valendo destacar:

\begin{itemize} 
\item Progressão de estágios \\
Em analogia a outros fenômenos biológicos, como reprodução ou ganhar habilidade de voar, a aquisição também evolui em seqüência. Por exemplo, produzir sons vem antes produzir palavras; produzir palavras antes de sentenças.  
\item Linha do tempo comum \\
Outra evidência é que diferentes crianças adquirem capacidades lingüísticas na mesma posição aproximada no tempo.
\item Período crítico \\
É observável que existe um período depois do qual a aquisição de linguagem se torna prejudicada. Crianças demonstram menos facilidade de aquisição de uma segunda língua após a primeira infância. Há inclusive casos extremos de privação total, em que se torna impossível adquirir fluência na língua pelo indivíduo, mesmo quando exposto mais tardiamente à linguagem. Essas observações sustentam a hipótese de que durante a primeira infância existem componentes biológicos em ação que são suprimidos no decorrer da vida.
\item Independência de estímulo externo \\
Observar que crianças surdas são capazes de produzir sons e se comunicar leva a crer que a linguagem é uma decorrência de um programa biológico e não exclusivamente dependente de estímulo externo. Vale lembrar que como outros fenômenos da natureza, é preciso estímulo para que se desenvolva por completo, por exemplo, sem uma nutrição mínima, o organismo não se desenvolve corretamente e até mesmo pode morrer. 

\end{itemize}

Se postos lado a lado, um bêbe humamo e um chimpanzé, assusta-se perceber que o segundo vai nos superar em muitas tarefas. Sobretudo, em se tratando de memorização. Incrivelmente, um ser humando, poucos anos depois, é capaz de dominar grande número de sentenças e conceitos. Os seres humanos parecem ter um instinto para linguagem como pássaros têm para voar e formigas para sociedade \citep{pinker94}.

\subsection{Teoria de Aprendizagem}

Aquisição da Linguagem é o processo pelo qual seres humanos tornam-se capazes de compreender e reproduzir uma língua. Esse processo, de um ponto de vista analítico, pode ser abordado a partir de questões de \emph{o quê}, \emph{quando} e \emph{como} um indivíduo desenvolve uma determinada habilidade lingüística \citep{pearl2010}. Então, seria possível organizar, de maneira simplificada, um fluxo lógico dos processos que levariam à aquisição de uma língua nas perguntas abaixo.

\begin{itemize}
\item Qual é o objeto a ser aprendido?
\item Quando o objeto é aprendido?
\item Como o objeto é aprendido?
\end{itemize}

Pensando na língua mãe, o objeto em questão são competências e conhecimento lingüísticos que a criança adquire, isto é, as palavras e os fonemas e, junto com eles, a carga semântica e regras sintáticas. O próximo passo é analisar o momento em que cada habilidade é dominada e estabelecer uma seqüência ordenada de acontecimentos, pontuando na linha do tempo marcas de proficiência com a língua. Por exemplo, a capacidade de discernir palavras na fala é obviamente anterior à capacidade de conjugar verbos no passado.

Contudo, analisar exatamente o desenrolar dessa etapas, isto é, \emph{como} a criança aprende \emph{o que} até o tempo de \emph{quando}, é mais intrincado. De todas as perguntas, o \emph{quando} é o mais bem resolvido, talvez por se tratar da parte mais empírica e observável das três. Se expostas a estímulos lingüísticos, crianças adquirem domínio sobre habilidades na linguagem em torno da mesma faixa de idade e com um conjunto similar de vocabulário. Além disso, o processo de aquisição se dá em etapas sucessivas e facilmente reconhecidas, como mencionado na seção \ref{sec:processobiologico}. Por exemplo, a identificação dos sons é anterior à combinação de palavras e a fala é anterior à escrita \citep{bertolo2001language}.

Agora a questão de \emph{o que}, isto é, o conhecimento lingüístico adquirido ganha um grau de complexidade maior. Como veremos na seção a seguir, não é possível que um indivídio seja exposto a todo conteúdo de uma língua, porém, toda criança é capaz de generalizar e aprender as regras da linguagem e passar a formar suas próprias sentenças. Existe grande esforço acadêmico nessa investigação, tanto por sua natureza científica quanto pela natureza prática e comercial devido ao uso que se pode ter em aplicacões.

A questão do \emph{como}, por fim, talvez seja a mais filosoficamente interessante e a mais resistente a métodos quantitativos. A capacidade de linguagem é uma característica absolutamente humana, o que leva a crer que deve existir alguma estrutura biológica exclusiva, visto que o cérebro humano não difere muito de tamanho de outros animais, inclusive proporcionalmente.

\subsubsection{Estímulo Positivo e Estímulo Negativo}

Todo estímulo lingüístico pode ser classificado como positivo ou negativo no sentido de ser uma forma válida ou não-válida dentro da língua em questão, isto é, se é gramatical ou agramatical. Uma criança exposta à fala de adultos é submetida a estímulos positivos na maior parte do tempo. Se recebe estímulos negativos, nunca é apresentada com explicações ou categorizações. Contudo, a criança é capaz de extrapolar a experiência e passar a formular suas próprias sentenças. Outro ponto é quando criança chega a inferir regras incorretas, ela muitas vezes não é corrigida, não tendo estímulos negativos. Apesar da falta de evidências negativas, a criança é capaz de evitar e eliminar erros de sua fala e consegue reproduzir a gramática de sua língua com o tempo.
A limitação de estímulo, tanto da insuficiência de evidência positiva quanto da falta de evidência negativa, deu origem ao argumento da  \emph{pobreza de estímulos} (do inglês, \textit{poverty of stimulus}). Esse é um argumento importante para a hipótese de que desenvolver um sistema complexo como a gramática de um adulto a partir de um conjunto limitado de informações deve pressupor uma capacidade inata para linguagem \citep{chomsky65}.

\subsubsection{Gramática Universal}

Em resposta ao argumento anterior, Chomsky propõe a Gramática Universal \citep{chomsky65}. Um falante é capaz de compreender e produzir um número aparentemente infinito de expressões lingüísticas, apesar do conjunto limitado de estímulos positivos e negativos ao qual é exposto. Não é possível pensar em um modelo indutivo que resolva esse paradoxo, o que leva a hipótese de uma dotação inata que permitiria a capacidade de uma atividade complexa como a linguagem.

\section{Lingüística Computacional}

Ao se atacar o problema de aquisição de linguagem, existem muitas formas de abordagem. O uso de ferramentas computacionais e modelos matemáticos contribuíram para a evolução desde sua introdução pelos cognitivistas. O principal benefício de uma abordagem computacional é que se trata de uma simulação, assim sendo possível ter controle e flexibilidade sobre as condições que se deseja experimentar \citep{frank2011}.

Ao se reproduzir o comportamento do ser humano em uma esquematizaçao ou em um computador, é importante esclarecer que não se faz necessário ter compreensão de todos processos em andamento. Segundo \citet{marr1982vision}, qualquer processo cognitivo pode ser dividido em três componentes:

\begin{itemize}
\item Computacional
\item Algorítmico
\item Implementacional
\end{itemize}

Na etapa computacional, tem-se o interesse de descrever o problema em questão, suas premissas e condições. Na etapa algorítmica, o centro da pesquisa é o modo como os processos de desenrolam. Por último, na etapa implementacional, tem-se um realização concreta do experimento. É nesse item que simulações computacionais trazem muitas contribuições, graças à possibilidade de controlar as entradas e saídas.

Ainda dentro de uma abordagem computacional, existem muitas vertentes com preposições, abordagens e modelos diferentes\citep{kaplan:inria-00348493}. A seguir, algumas serão apresentadas.

\begin{itemize}
\item Generativa \\
Essa é a visão que mais dominou o estudo da área desde o surgimento. Segundo ela, a aquisição de linguagem consiste no aprendizado da sintaxe, isto é, a função de classificar sentenças em gramaticais e agramaticais. Computacionalmente, o objetivo é de mapear um espaço de sentenças $S$ nas respostas \emph{gramatical} ou \emph{agramatical}. Esse é um problema muito parecido com aprendizado estatístico ou de máquina, em que existe um espaço de exemplos para treinamento e uma classificação no final. O que difere essa abordagem fortemente de um problema de classificação usual é a hipótese de pobreza de estímulos, ou \textit{poverty of stimulus}, segundo a qual nunca seria possível construir um modelo completo da gramática considerando-se a limitação dos estímulos. Nesse ponto de vista, o interesse maior está em construir a competência lingüística a partir de amostras.

\item Estatística \\
A hipótese de pobreza de estímulos, ou \textit{poverty of stimulus}, foi colocada em questão em experimentos mostrando ser possível extrair padrões lingüísticos de sons e sentenças. Em contraste com o item anterior, o interesse maior está em modelar a performance lingüística. Essa abordagem é a mais aberta para as técnicas estatatísticas e de aprendizado de máquina, fazendo um algoritmo capaz de avaliar sentença um problema de classificação por excelência. Como principal desvantagem está que só é possível aprender estruturas com grande ocorrência na língua. Também se faz necessário o uso de grandes \emph{corpora} para que os resultados tenham relevância. De qualquer modo, é uma das principais abordagens adotadas por programadores quando se trata de problemas envolvendo linguagens naturais.

\item Socia e Espacial \\
É importante ressaltar que muitos modelos matemáticos e computacional ignoram a componente social na aquisição de linguagem. Contudo, é claro que tanto o aprendizado quanto a expressão de uma língua têm forte peso da sociedade e meio em que se está inserido. Essas abordagens têm grande contribuição para o entedimento do contexto e semântica da fala. Por exemplo, se durante um processo de aprendizado de máquina existir um contexto de ação, é muito mais fácil fazer inferências e diminuir ambigüidades.

\item Desenvolvimentista \\
As abordagens anteriores descrevem o processo de aprendizado como contínuo e determinístico, em que o falante tem pouco poder de ação, isto é, o indíviduo seria somente exposto a estímulo de outros indíviduos ou do ambiente e passaria a adquirir graças a uma faculdade inata. Alguns linguistas acreditam que a aquisição de linguagem também se deve a maneira de explorar o conhecimento lingüístico de uma forma não-linear e não-determinística. Para eles, todo processo é desenvolvido progressivamente ao se interagir e com o modo de se interagir com outras pessoas e outras ideias.

\end{itemize}

\subsection{Abordagem com aprendizado de máquina}

Em 1997, um ser humano foi vencido por um computador em uma partida de xadrez\footnote{\url{http://www-03.ibm.com/ibm/history/ibm100/us/en/icons/deepblue/}}. O super computador Deep Blue superou somente com circuitos e algoritmos um dos limites de inteligência humana. Porém, vale lembrar que foram necessários mais de cinqüenta anos de pesquisa, engenho e sorte. Mas, no final, uma máquina superou o ser humano. Em 2017, Google realizava testes com um veículo autônomo\footnote{\url{http://www.google.com/selfdrivingcar/}}, isto é, sem a necessidade de um motorista humano. Novamente, fomos vencidos.

Agora, basta ter uma conversa com a Siri ou a Alexa para perceber que ninguém jamais aceitaria um convite para jantar de qualquer uma delas. Embora rendam boas piadas, os seus mecanismos de fala são muitos artificiais. O que torna a capacidade de falar tão inerentemente humana? E o que torna falar tão intricamente difícil? Aliás, uma criança de 5 anos é capaz de falar com naturalidade, apesar de não fazer ideia do que fazer com uma alavanca do câmbio de marchas.

Embora as técnicas de Inteligência Artificial tenham evoluído muito e novas capacidades computacionais tenham surgido, a linguagem ainda tem um grau de complexidade resistente a uma apreensão plena pelos métodos da área. A maioria das soluções de aprendizado de máquina fazem uso de propriedades estatísticas nas massas de dados e extrapolam padrões. Porém, tudo indica que a linguagem não é uma mera repetição de padrões e que deve existir um instinto ou uma habilidade inata para adquirir uma língua.

De qualquer modo, já existem muitos exemplos de aplicações com linguagens naturais usando-se técnicas estatísticas, como corretores de texto, transcrições automáticas de fala e tradutores.